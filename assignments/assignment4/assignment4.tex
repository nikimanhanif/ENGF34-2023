\documentclass{article}

\usepackage{graphicx}
\usepackage{amsmath}

\title{ENGF34 Design \& Professional Skills\\
  Assignment 4: implement a Tetris autoplayer}
\author{}
\date{}

\begin{document}

\maketitle

\section*{Instructions}

In this assignment, you will write an AI that plays Tetris. You will
push your code to a Git repository, where a server will run it and
keep track of your best score.

To get a passing grade (40\%) all you need to do is beat the random
autoplayer you’ll be given. Everyone should be able to do this. To get
100\%, you also need to beat everyone else in the year. You grade will
be determined by the median score your code achieves from multiple
runs with different random block sequences.

To start, perform the following steps:
\section{Register for a GitLab account}
  
If you have not done so already, register for an account on UCL's
GitLab instance (\texttt{https://scm4.cs.ucl.ac.uk}). You need to use your
\texttt{@ucl.ac.uk} email address for this. Please use your real name when
registering; otherwise, we will not know which submissions are yours!

\section{Register on the web interface}

The web interface that we will be using can be found on
\begin{verbatim}
https://engf0002.cs.ucl.ac.uk
\end{verbatim}

Our apologies for not updating the server name to engf0034!

Go to that page, and click the button
that reads “Sign in with GitLab”. You will be redirected to a page
that asks you to grant API access to the grading server; click
“Authorize”.

\section{Fork the boilerplate code}

If all went well, you have been redirected back to the web
interface. Under the heading “Available contests”, there should be an
entry labelled “Tetris”. Click the button that says “Enrol”. This will
create a copy (fork) the Tetris boilerplate repository into your
GitLab account.

\section{Clone your fork}
  
To pull in the code, you need to install Git if you haven't already done so;
refer to  \texttt{https://git-scm.com/} for instructions on how to install Git on your
system. After installation, you can pull in your code by running:
\begin{verbatim}
git clone https://scm4.cs.ucl.ac.uk/YOUR_GITLAB_USERNAME/tetris.git
\end{verbatim}

\noindent(substitute your actual username for \texttt{YOUR\_GITLAB\_USERNAME} in this command!)

You will be prompted for your GitLab username and password. If this
worked, you should have a folder named tetris in your working
directory on your computer. Enter this folder by typing \texttt{cd tetris}

\section{Run the code}
  
You should be able to use any recent version of Python 3. The auto-grader is
currently running python 3.7.9, but you should be able to run more
recent versions, so long as you don't use any very new language features.

\textbf{Note: If you've installed python as \texttt{python3} on your computer, substitute \texttt{python3} for python and \texttt{pip3} for pip in all the instructions below.}

Once you are in the code directory, you should be able to start the interface by running:
\begin{verbatim}
    python visual.py
\end{verbatim}
To get a feeling for the game, you can also run any of the interfaces in manual mode by adding the flag -m:
\begin{verbatim}
    python visual.py -m
\end{verbatim}
If that interface does not work, there are two alternative interfaces available:
\begin{itemize}   
\item A command-line interface; run \texttt{python cmdline.py} to use
  it. If you are using Windows, you may need to install the
  \texttt{windows-curses} package first; try running:
\begin{verbatim}
    pip --user install windows-curses.
\end{verbatim}
\item A Pygame-based interface; run python visual-pygame.py to use it. For this, you need to have a working
  copy of Pygame; run:
\begin{verbatim}
    pip --user install pygame
\end{verbatim}
to install it.
\end{itemize}

The pygame version is recommended as it works fastest when autoplaying.

If you get an interface running, you should see the default player in action. This player is not very clever: it chooses random actions in hopes of eliminating a line. It is your job to write a better AI.
 
\section{Our Tetris Variant}

In the regular version of Tetris blocks of four squares known as
Tetronimos drop, and you must guide them to a good landing place. When
one or more lines is completed all the blocks in those lines are
eliminated, the blocks above drop, and you score extra points. The
goal is to score as many points as possible before the blocks reach
the top of the screen. Normally the game gets faster as it progresses,
so gets harder.

Our version has a number of differences from normal Tetris:
\begin{itemize}
\item If you play it manually, it doesn’t get faster. This is because, within limits, we don’t want the amount of time you take to calculate a move to matter.
\item You only have 400 tetronimos to land. This changes the gameplay significantly. Just staying alive to the end is not going to result in a really high score.
\item The scoring is weighted heavily in favour of eliminating more rows at a time. The scores for eliminating 1-4 rows are 25, 100, 400, 1600, so there’s a big benefit from not eliminating one row at a time.
\item You can discard a tetronimo you don’t like. You only get to do this 10 times though. This makes play easier for a human, but it’s not necessarily easy for an AI to decide to do this.
\item You can substitute a bomb in place of the next tetronimo! When a bomb lands it destroys all the immediately surrounding blocks, and all the blocks above it or the neighbouring squares will fall to the ground. Blocks destroyed by bombs don’t score anything. You only get five bombs, so use them wisely. Bombs also make play easier for a human, but add a new element for an AI.
\end{itemize}
\subsection*{Keybindings}

In manual play, the key bindings are:
\begin{description}
\item{↑} - rotate clockwise
\item{→} - move right
\item{←} - move left
\item{↓} - move down
\item{Space} - drop the block
\item{z} - rotate anticlockwise
\item{x} - rotate clockwise
\item{b} - switch next piece for bomb d - discard current piece
\item{Esc} - exit game
\end{description}
If you don’t like the key bindings, feel free to change them. They’re
in key to move in visual-pygame.py, UserPlayer.key() in visual.py, or
UserPlayer.choose action() in cmdline.py.

\section{Write your player}
Your code should be placed in the file \texttt{player.py}. A player is a class with a function called choose action. This function will be called with a copy of the Tetris board in the variable board every time an action by the player is required. The choose action function you write should decide what actions to take based on the state of the board. To this end, you can use the following information:
\begin{itemize}
\item The variable \texttt{board.falling} contains information about the currently falling block; most importantly, its shape (\texttt{board.falling.shape}) and the coordinates of its cells (\texttt{board.falling.cells}). The shape is an instance of the enum \texttt{shape}, whose definition can be found in \texttt{board.py}.
\item The variable \texttt{board.next} contains information about the
  next block that will fall. The cells do not mean anything yet (the
  block is not on the board), but you can check out the shape. It is
  possible that \texttt{board.next} may be \texttt{None} if you’re testing moves in the
  sandbox (see below) and the first block has already landed.  
\item The property \texttt{board.cells} contains information about the
  currently occupied cells on the board. More precisely, this property
  is a set of tuples (x, y) that signify locations. Note: the origin
  of the board is in the top-left corner; the x-axis extends to the
  right, and the y-axis extends downwards.

You can iterate over a set the same way you might iterate over a list; for instance:
\begin{verbatim}  
    for (x, y) in board.cells:
        # Do something here.
\end{verbatim}

In the above, the order of cells is not guaranteed. One thing you may
want to do, as you analyse the board, is iterate over the cells
already occupied from top to bottom, left to right. This can be done
as follows:
\begin{verbatim}  
    for y in range(board.height):
        for x in range(board.width):
            if (x, y) in board.cells:
                # Do something here.
\end{verbatim}
The board also has five methods that can be used to try out the effect of actions:
\item The method \texttt{move()} accepts an instance of the enum
  \texttt{Direction}, whose definition can be found in board.py. For
  instance, to try moving the block to the left, use
  \texttt{board.move(Direction.Left)}. The instance of board will then
  be updated with the effect of that move. To drop the block, use
  \texttt{Direction.Drop}.
\item The method \texttt{rotate()} accepts an instance of the enum
  \texttt{Rotation}; again, refer to \texttt{board.py} for the
  definition. For instance, to rotate a block anticlockwise, use
  \texttt{board.rotate(Rotation.Anticlockwise)}.
\item The method \texttt{skip()} simulates the effect of skipping a
  move; it does not take any arguments.
\item The method \texttt{discard()} simulates the effect of discarding
  the current block; it does not take any arguments.  If you have no
  discards left the call will behave the same as skip().
\item The method \texttt{bomb()} simulates the effect of switching the
  next block for a bomb; it does not take any arguments. If you have
  no bombs left, or if \texttt{board.next} is \texttt{None}, the call
  will behave the same as \texttt{skip()}.
\end{itemize}
Calling any of the above (including skip) will also apply the implicit
move down that occurs once every turn. Furthermore, if the action (in
the case of a move) or the subsequent implicit move down caused the
block to land, these functions will return \texttt{True}; otherwise,
they will return \texttt{False}.

To experiment with multiple options, it is useful to get a copy of the
board and try the effects of your actions there; actions on the copy
will not affect the original, so you can use that state later:
\begin{verbatim}
sandbox = board.clone()
sandbox.rotate(Rotation.Anticlockwise)
sandbox.move(Direction.Drop)
sandbox.move(Action.Bomb)
# Check if the new board in sandbox is to your liking
\end{verbatim}

Once your player is ready to commit to a action, you can tell the
interface by returning that action. For instance, you can use
\begin{verbatim}
return Rotation.Clockwise
\end{verbatim}
to tell the interface that you want to rotate the
block clockwise, or
\begin{verbatim}
return Action.Discard
\end{verbatim}
 to tell the interface to
discard the current block. To skip a move, use the special value None,
i.e., write \texttt{return None}.

If your AI has a plan of multiple action you can either return those as a list, i.e.,
\begin{verbatim}
return [Rotation.Anticlockwise, Direction.Left, Direction.Drop, Action.Bomb]
\end{verbatim}
or you can use the yield statement to return each individual action:
\begin{verbatim}
yield Rotation.Anticlockwise yield Direction.Left
yield None
yield Direction.Drop
yield Action.Discard
\end{verbatim}
The latter also works in loops, and can be a nice way to synthesize a
plan based on your earlier explorations without first having to build
a list and return that. If you’re not comfortable with Python’s \texttt{yield}
command, it’s fine to just return actions.

Make sure you don't return an empty list of moves - this can result in
your code getting into an infinite loop!

If your player returns multiple actions, those actions will only be
executed until the currently falling block lands. After that, the
system will disregard any remaining moves from your list, generate a
new block, and your function choose action will be called again, with
the new board containing the new falling piece (what was previously
\texttt{board.next}) as well as the new next piece. On the other hand,
if the moves supplied by your player run out before the block lands,
the function choose action will be called again as well.

\section{Push your code}
Once you are happy with your player, you can commit your changes to
the repository. First, stage your changes by running git add, e.g., to
stage your changes to player.py, run:
\begin{verbatim}
    git add player.py
\end{verbatim}
Next, you commit your changes by running:
\begin{verbatim}
    git commit
\end{verbatim}
This will bring up an editor asking for a commit message, where you
describe your changes. Once you save and close the editor, the commit
will be made. You can also use the -m flag to type your message on the
command line, e.g.:
\begin{verbatim}
   git commit -m "Fixed all the bugs.".
\end{verbatim}

Finally, you can upload (push) your code to GitLab by running:
\begin{verbatim}
    git push
\end{verbatim}
You will again be prompted for your GitLab username and password
when you do this. When your code is pushed, the server will run it on
some (secret) inputs.

You should be able to track the score of your submission in real time
through the web interface on
\texttt{https://engf0002.cs.ucl.ac.uk/submissions/tetris}

Near to the deadline, the grading server will get very busy (expect delays of hours!). The grading
server will grade your submissions in the order you push them, and
will use a fair queuing algorithm to ensure that you don’t get more
than your fair share of submissions when it is busy. Fair queuing
means that when it wants to choose a submission to run, it will look
at all users with queued submissions; the next one it will run is from
the user whose last submission it ran was the longest ago. If you keep
modifying your code and keep pushing, it may take a long time to get
to your most recent submission. If you no longer care about the result
(because, for example, you just fixed a bug), consider cancelling old
submissions (via the submissions page) before they are picked up to
run.

\subsection*{Caveat: changing files and state}
You should not have to change anything but \texttt{player.py}, but
modifying other files or adding files is not prohibited. In the best
case, any changes to the game logic only have an effect locally — the
test environment maintains its own copy of the game logic; in the
worst case, your changes may crash the code as it is
evaluated. Similarly, the test environment maintains its own copy of
the board based on the actions that your player takes; the player will
not be able to access this state. This means that, for instance,
changing \texttt{board.score} to a value of your liking will not
change the score maintained by the server.

\subsection*{Caveat: using python libraries}

The grading server has a vanilla verion of python installed, with no
additional libraries.  If you include any external libraries in your
code, you will find that your solution will crash when run on the
grading server.  You must write your solution code yourself without
using any additional libraries.  You don't need to use any additional
libraries to do well on this assignment.

If you're considering an advanced solution and want to use external
code to train weights in your program, this is permitted, but that
code must not be included in your player.py file.  Talk to us if
you're thinking of doing this, just to check that what you're
considering is in the spirit of the assignment.

You are not permitted to use AI assistance such as ChatGPT to help
your write your code.  You should also not search online for code that
plays Tetris.  In any event, our version of limited-moves-Tetris is
sufficiently different from regular Tetris that solutions found online
will not score especially well.

\subsection*{Optional: generate a keypair}
Typing your username and password for every push can get cumbersome
after a while. You can get around this by generating a private and
public key, and adding the public key to your GitLab account. First,
generate your keypair by running
\begin{verbatim}
   ssh-keygen -t rsa -b 2048
\end{verbatim}
This may ask for a passphrase for the key; you can leave it empty if
you like, or encrypt the key with your passphrase for extra
security. To add the public key to GitLab, first copy the output of
the following command:
\begin{verbatim}
   cat ~/.ssh/id_rsa.pub
\end{verbatim}
Next, go to your SSH keys on on GitLab (click the icon in the
upper-right corner, click “Settings”, click “SSH Keys” in the menu on
the left) and paste the public key in the field labelled “Key”. Once
you give the key a title, you can click “Add key”. Finally, we need to
adjust the way git talks to GitLab, by running
\begin{verbatim}
   git remote set-url origin git@scm4.cs.ucl.ac.uk:YOUR_GITLAB_USERNAME/tetris.git
\end{verbatim}
You should now be able to push and pull without giving your
credentials (but possibly typing the passphrase for your key if you
have not left it empty when generating it).

\subsection*{A word on fairness}
To ensure that all submissions get evaluated on a level playing field,
the server will evaluate your code in a limited environment. In
particular, this means that you are given a limited amount of memory
and CPU usage to make your calculations. We don’t expect you will hit
these limits with a reasonable algorithm. Moreover, your code will not
be able to write to the directory where it is running, and it will
also not be able to communicate over the network.

Your code will also be given a limited number of blocks and a limited
amount of time to run in, so your aim is not to play forever, but to
achieve a good score in the time you have. If your code runs out of
time or blocks, it will be stopped, but the score will still
count. However, if your code runs out of memory or crashes for some
other reason, its score will be kept but the submission will not be
counted.  We have made every effort to make sure that these
limitations are obeyed; if you somehow find a way to circumvent them,
we would like to hear it. The real challenge, however, should be to
write a great autoplayer! 
\end{document}
